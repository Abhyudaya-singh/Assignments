\documentclass[12pt]{article}
\usepackage[english]{babel}
\usepackage{natbib}
\usepackage{url}
\usepackage[utf8x]{inputenc}
\usepackage{amsmath}
\usepackage{graphicx}
\graphicspath{{images/}}
\usepackage{parskip}
\usepackage{fancyhdr}
\usepackage{vmargin}
\setmarginsrb{3 cm}{2.5 cm}{3 cm}{2.5 cm}{1 cm}{1.5 cm}{1 cm}{1.5 cm}

\title{MEDICAL DEVICES}					
\author{21111004}								
\date{28 JAN 2022}								

\makeatletter
\let\thetitle\@title
\let\theauthor\@author
\let\thedate\@date
\makeatother

\pagestyle{fancy}
\fancyhf{}
\rhead{\theauthor}
\lhead{\thetitle}
\cfoot{\thepage}

\begin{document}
\begin{titlepage}
	\centering
    \includegraphics[scale = 0.20]{logo.jpg}\\[1.0 cm]	
    \textsc{\LARGE National Institute Of Technology \newline\\\\ RAIPUR}\\[2.0 CM]
    
	\textsc{\Large ASSIGNMENT 01}\\[0.5 cm]				% Course Code
	\rule{\linewidth}{0.4 mm} \\[0.4 cm]
	{ \huge \bfseries \thetitle}\\
	\rule{\linewidth}{0.4 mm} \\[1.5 cm]
	
	\begin{minipage}{0.6\textwidth}
		\begin{flushleft} \large
			\emph{Submitted To:}\\
			Saurabh Gupta\\
            Department Of Basic Biomedical Engineering\\
			\end{flushleft}
			\end{minipage}~
			\begin{minipage}{0.4\textwidth}
            
			\begin{flushright} \large
			\emph{Submitted By :}\\
			Abhyudaya Kumar Singh\\
            21111004\\
		\end{flushright}
        
	\end{minipage}\\[2 cm]
\end{titlepage}

\tableofcontents
\pagebreak

\section{Automated External Defibillator}
An \textbf{ \emph{Automated External Defibrillator  }} is a lightweight, portable , life saving decice  device that delivers an electric shock through the chest to the heart to treat people experiencing \textbf{ \emph{Sudden Cardiac Arrest (SCA)}
}, a medical condition in which the heart Malfunctions and stops beating suddenly and unexpectedly ( If not treated within minutes, it quickly leads to death).\newline
 This shock momentarily stuns the heart and stops all activity. It gives the heart the chance to resume beating effectively. AEDs advise a shock only for ventricular fibrillation or another life-threatening condition called \emph{ pulseless ventricular tachycardia}.\newline
 Most SCAs result from \emph{ Ventricular fibrillation (VF)}. VF is a rapid and unsynchronized heart rhythm that originates in the heart’s lower chambers (the ventricles). The heart must be \emph{“defibrillated”} quickly, because a victim’s chance of surviving drops by seven to 10 percent for every minute a normal heartbeat isn’t restored.\newline \\
 The AED system includes accessories, such as a battery and pad electrodes, that are necessary for the AED to detect and interpret an electrocardiogram and deliver an electrical shock. A built-in computer checks a victim’s heart rhythm through adhesive electrodes. The computer calculates whether defibrillation is needed. \newlinw
AEDs can be semi-automated or fully automated:- \newline
\begin{itemize}
\item  \textbf{\emph{Semi-Automated Defibrillators}} analyze the heart's rhythm, and if an abnormal heart rhythm is detected that requires a shock, then the device prompts the user to press a button to deliver a defibrillation shock. \newline
\item \textbf{\emph{Fully Automated Defibrillators}} analyze the heart's rhythm and deliver a defibrillation shock if commanded by the device software without user intervention.
\end{itemize}
\newline\\

AEDs are safe to use by anyone . Non-medical personnel such as police, fire service personnel, flight attendants, security guards and other lay rescuers who have been trained in CPR can use AEDs. However, AEDs are intended for use by the general public. Most AEDs use audible voice prompts to guide the user through the process.\newlinw
 Some studies have shown that 90 percent of the time AEDs are able to detect a rhythm that should be defibrillated. This data suggests that AEDs are highly effective in detecting when (or when not) to deliver a shock.\newline \\
 All first-response vehicles, including ambulances, lawenforcement vehicles and many fire engines should have an AED. AEDs also should be placed in public areas such as sports venues, shopping malls, airports, airplanes, businesses, convention centers, hotels, schools and doctors’ offices. They should also be in any other public or private place where large numbers of people gather or where people at high risk for heart attacks live. They should be placed near elevators, cafeterias, main reception areas, and on walls in main corridors.\newline \\\\
 \begin{figure}
     \centering
     \includegraphics[scale = 0.25]{AED.jpg}
     \caption{Automated External Defibrillator}
 \end{figure}
 \newpage
 \section{SPECT/CT}
\textbf{\emph{Single photon emission computed tomography/computed tomography\newline (SPECT/CT)}} combines two different diagnostic scans into one for a more complete view of the body region being studied.\newline The SPECT scan uses nuclear medicine to give good images of metabolic abnormalities, whereas the CT scan may be able to help narrow down specifically where the problem is occurring, such as in the bone or nearby tissue.\newline\\
A SPECT/CT scan typically involves 3 main components:
\begin{enumerate}
\item A \textbf{\emph{Radioactive Tracer}} is injected into the body’s bloodstream. This tracer can be seen by a gamma scanner, which shows metabolic functions of tissues and organs, such as blood flow and potential abnormalities in the tissues.
\item A \textbf{\emph{CT}} scan is taken with an \emph{ X-ray scanner} rotating around the body region being studied.
\item A \textbf{\emph{SPECT }}scan is taken with a \emph{gamma scanner} rotating around the body region being studied, which takes much longer than the CT scan.
\end{enumerate}\\
Since the SPECT and CT scans both form cross-section images of the same areas, the computer is able to combine these images. The resulting cross-section images show the x-ray images of the bones overlaid with the nuclear imaging.\newline 
A SPECT/CT scan may be particularly useful when trying to get a look at a metabolic abnormality, as well as its location in relation to the bone. For example, SPECT/CT may show that an abnormality suspected of causing back pain is actually located in the facet joint rather than the vertebral body or another part of the spine. Another example would be to see how much of a cancerous tumor has gone into the bone.\newline \\
While SPECT/CT is considered relatively safe, this combined scan tends to use more radiation than other scans.\newline 
 If you receive an injection or infusion of radioactive tracer, you may experience:
 \begin{itemize}
\item Bleeding, pain or swelling where the needle was inserted in your arm.
\item Rarely, an allergic reaction to the radioactive tracer.
\end{itemize}
Most of the radioactive tracer leaves your body through your urine within a few hours after your SPECT scan. Your doctor may instruct you to drink more fluids, such as juice or water, after your SPECT scan to help flush the tracer from your body. Your body breaks down the remaining tracer over the next few days.\newline
SPECT scans aren't safe for women who are pregnant or breast-feeding because the radioactive tracer may be passed to the developing fetus or the nursing baby.\newline \\
 Pictures from your scan may show colors that tell your doctor what areas of your body absorbed more of the radioactive tracer and which areas absorbed less. For instance, a brain SPECT image might show a lighter color where brain cells are less active and darker colors where brains cells are more active. Some SPECT images show shades of gray, rather than colors.\newline \\
 \begin{figure}
     \centering
     \includegraphics[scale =2.00]{SPECT-CT.png}
     \caption{SPECT/CT Scanner}
 \end{figure}
 \newpage
\section{Telemedicine System}
\textbf{\emph{Remote Healthcare}} machinery such as \textbf{\emph{ Telemedicine  System }}and \textbf{\emph {Remote Monitoring}} are the potential solution to the problems of rural health care which suffers from the lack of medical infrastructure and medical professionals.\newline\\
 With the widely dispersed populations, efficient medical services and health care have scarcely touched the lives of the people in rural areas. India, being the second-most populated nation in the world with a total area of \emph{3.28 million} Sq. Km., sustains \emph{16.7 percent }  of the world population with over \emph{1.35 billion} residents . Around \emph{70 perecent}  of India's population is residing in rural areas in which nearly \emph{800 million} people living in the residential area on the outskirts of a city have no direct access even to primary health care. The ready availability of health care services still has not kept pace for these low-income and under-served individuals in the suburbs.\newline\\
 Telemedicine  system provides quality medical services by bringing the knowledge and experience of health professional closer to patients with less healthcare cost. The four categories of telemedicine systems are :-
 \begin{enumerate}
 \item Asynchronous Telemedicine 
 \item Synchronous Video Conferencing or Interactive telemedicine
 \item Remote Patient Monitoring 
 \item Mobile Health or m-health
 \end{enumerate}\\
Telemedicine systems, in general, follow a hierarchical tiered structure which includes the following:
\begin{itemize}
\item Level 1: Local/remote telemedicine center: These are the local or primary healthcare unit located in rural and remote areas.
\item Level 2: City/district hospital:Local/rural health centers are connected to the city/district hospital. The district hospital, optionally, may further be connected to the state hospital.
\item Level 3: Speciality center :The city hospital is connected to the speciality centers for disease-specific further assistance.
\end{itemize}\\
A patient requiring medical attention approaches the nearby local health center where a local health professional (may not be a certified doctor) attends the patient and does the primary health check-up. This unit consists of basic diagnostic equipment and tele-consultation devices linked via PC and Internet to the city hospital. The primary responsibility of the local healthcare unit is to acquire all the vital statistics of the patient in terms of physiological data (e.g., blood, urine, etc.) and images (e.g., ultrasound) and transmits the data to the remote city hospital.\newline\\
After receiving the records, the remote medical practitioner goes through every detail, before proceedings with live Interaction with patients. After carefully examining the basic vital signs, the meeting is booked online between doctor and patient at remote healthcare unit. The doctor makes use of an audio or video conferencing system as well as automation live feeds to have live interaction with the patient. \newline \\
 These remote hospitals are connected to a centralized database where all the data of the patient as well as other details and even the recorded audio/video interaction between doctor and patient are also stored. The stored information can be accessed using mobile apps or web-based interface. The main hospitals are also linked to specialist hospitals to provide specialized support to the patients in case of an emergency and these specialized hospitals have same teleconferencing units enabled to support remote patients.\newline \\
 \begin{figure}
    \centering
    \includegraphics[sacle = 0.50]{telemedecine.jpg}
    \caption{Telemedecine System}
 \end{figure}
 \newpage
\section{Resuscitator}
A \textbf{\emph{ Resuscitator}} is a device used by individuals and health professionals to force  \emph{ oxygen} into the lungs of a person who is not breathing.\newline
\textbf{\emph{Manual Resuscitators} }require the use of physical exertion by the rescuer who tries to get the patient breathing again. \textbf{\emph{Gas-powered resuscitators}}, on the other hand, require little exertion on the part of the operator, whose main focus is to make sure that the unit does not malfunction and delivers the right amount of pressure.\newline\\
A manual resuscitator is portable, and while some are primarily used by emergency medical professions, they can be included in medicine cabinet or first aid kit for use by lay persons. The two main types of manual resuscitators are the \emph{Bag Valve Mask (BVM)}, which is commonly used by those in the medical profession, and the \emph{breath-powered resuscitator}, often used by individuals who are not in a medical profession.A manual resuscitator should be used on a victim only in an environment where the air is unquestionably safe to breathe.\newline\\
The three components of a BVM resuscitator are a bag, a mask, and a valve. The bag resembles a bulb, and is squeezed to ventilate the patient with ambient air instead of oxygen from a gas powered pressure tank. The mask goes over the patient's face to prevent air from escaping and to help channel the oxygen into the lungs. The valve controls the rate of flow of air into the lungs. An emergency service professional can transform a BVM to a gas-powered one by connecting the resuscitator to a tank.\newline \\
A breath-powered resuscitator consists of a facial mask that goes over the patient's nose and mouth, with a tube that protrudes out for a rescuer to breathe oxygen directly into the patient's lungs.  Some do not have a mask at all, and the rescuer simply inserts a wide tube into the mouth and breathes air into the patient. This is the least effective option, because the air is not trapped by a mask. Breath-powered resuscitators are easy to use, however, and are usually inexpensive. They do not have any bags to squeeze, and therefore there is less concern about fatigue.\newline \\
Gas-powered resuscitators provide oxygen to patients who are not breathing by the use of gas instead of human exertion. A person has to manually trigger the resuscitator by pushing a button or using a lever, but the unit containing oxygen does the work of delivering oxygen through a mask and endotracheal tubes. Many resuscitators have a "demand mode" feature, which automatically delivers oxygen based on how the patient is breathing. It is often better to use gas-powered resuscitators to avoid the fatigue associated with manual resuscitators, where the operator has to squeeze the bag over and over again while holding the mask in place. There is a risk of causing serious injury to the patient if a gas powered resuscitator malfunctions, however, and the pressure of the oxygen delivered is not limited.\newline\\
 In 1992 the Genesis(R) II time/volume cycled resuscitator (now upgraded to meet the current, International, resuscitation guidelines and called the CAREvent(R) ALS and CA)provide the SIMV automatic ventilation mode with demand breathing for the spontaneously breathing patient. These devices work like full blown transport ventilators yet are simple enough to operate that they can be used in an emergency situation by pre-hospital healthcare providers and are small enough to be easily transportable. Having a manual override control for use during mask CPR they meet the requirements of the current resuscitation standards. The Oxylator (R) EM-100 introduced in the late 1990s and subsequently replaced by the more flexible Oxylator (R) EMX and HD are pressure cycled devices that utilize pressure, rather than time, cycling to ventilate the patient. More recently the microVENT resuscitator range introduced two new models, the microVENT(R) CPR and the microVENT(R)World. These two new time/volume resuscitators meet the latest requirements for resuscitation and are lighter and smaller[5] than most competitor products.
\begin{figure}
    \centering
    \includegraphics[scale = 0.15]{manual resuscitator.png}
    \caption{Manual Resuscitator}
    \includegraphics[scale =0.25]{Gas powered resuscitator.png}
    \caption{Gas Powered Resuscitator}
\end{figure}
\newpage
\section{PACS}
\textbf{\emph{Picture Archiving and Communication System (PACS)}} is a medical imaging technology used primarily in healthcare organizations to securely store and digitally transmit electronic images and clinically-relevant reports. The use of PACS eliminates the need to manually file and store, retrieve and send sensitive information, films and reports. Instead, medical documentation and images can be securely housed in off-site servers and safely accessed essentially from anywhere in the world using PACS software, workstations and mobile devices. \newline \\
Medical imaging storage technologies such as PACS are increasingly important as the volume of digital medical images grows throughout the healthcare industry and data analytics of those images becomes more prevalent. \newline \\
PACS has four major components: 
\begin{enumerate}
\item  Hardware imaging machines
\item A secure network for the distribution and exchange of patient images 
\item A workstation or mobile device for viewing, processing and interpreting images and 
\item Electronic archives for storing and retrieving images and related allows for documentation and reports. 
\end{enumerate} \\
In turn, PACS has four main uses. The technology: 
\begin{enumerate}
\item replaces the need for hard-copy films and management of physical archives.
 \item allows for remote access, enabling clinicians in different physical locations to review the  same data simultaneously.
\item  offers an electronic platform for images interfacing with other medical automation systems such as a hospital information system (HIS), electronic health record (EHR), and radiology information system (RIS).
 \item allows radiologists and other radiology and medical personnel to manage the workflow of patient exams
 \end{enumerate} \\
Cloud-based PACS store and back up an organization's medical imaging data to a secure off-site server. \newline 
Although PACS processes are widely adopted in healthcare, \textbf{\emph{vendor neutral archive (VNA) }}technology has replaced PACS in some healthcare settings and integrates with PACS in others.\newline \\
PACS vendors employ various syntaxes within \emph{DICOM}, which makes it hard for data from one system to work in another system. VNAs enable data integration by deconstructing data from an originating PACS and then migrating the data to the new system with the proper syntax.\newline \\
DICOM enables imaging technologies to connect with and transfer health data to systems at other healthcare organizations. A RIS, a networked software system for managing medical imagery and associated data, is often used with PACS and VNAs to manage image archives, image orders, record-keeping and billing.
\begin{figure}
    \centering
    \includegraphics[scale=0.50]{PACS.png}
    \caption{PACS Software Network}
\end{figure}
\newpage


\end{document}