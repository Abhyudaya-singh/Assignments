\documentclass[12pt]{article}
\usepackage[english]{babel}
\usepackage{natbib}
\usepackage{url}
\usepackage[utf8x]{inputenc}
\usepackage{amsmath}
\usepackage{graphicx}
\graphicspath{{images/}}
\usepackage{parskip}
\usepackage{fancyhdr}
\usepackage{vmargin}
\setmarginsrb{3 cm}{2.5 cm}{3 cm}{2.5 cm}{1 cm}{1.5 cm}{1 cm}{1.5 cm}

\title{Evolution Of Modern Healthcare \newline  System}					
\author{21111004}								
\date{28 JAN 2022}								
\makeatletter
\let\thetitle\@title
\let\theauthor\@author
\let\thedate\@date
\makeatother

\pagestyle{fancy}
\fancyhf{}
\rhead{\theauthor}
\lhead{\thetitle}
\cfoot{\thepage}

\begin{document}
\begin{titlepage}
	\centering
    \includegraphics[scale = 0.20]{logo.jpg}\\[1.0 cm]	
    \textsc{\LARGE National Institute Of Technology \newline\\\\ RAIPUR}\\[2.0 CM]
    
	\textsc{\Large ASSIGNMENT 02}\\[0.5 cm]				% Course Code
	\rule{\linewidth}{0.4 mm} \\[0.4 cm]
	{ \huge \bfseries \thetitle}\\
	\rule{\linewidth}{0.4 mm} \\[1.5 cm]
	
	\begin{minipage}{0.6\textwidth}
		\begin{flushleft} \large
			\emph{Submitted To:}\\
			Saurabh Gupta\\
            Department Of Basic Biomedical Engineering\\
			\end{flushleft}
			\end{minipage}~
			\begin{minipage}{0.4\textwidth}
            
			\begin{flushright} \large
			\emph{Submitted By :}\\
			Abhyudaya Kumar Singh\\
            21111004\\
		\end{flushright}
        
	\end{minipage}\\[2 cm]
\end{titlepage}

\tableofcontents
\pagebreak

\section{Abstract}
Health Care System is constantly changing as a result of technological advancements, ageing populations, changing disease patterns, new discoveries for the treatment of diseases and political reforms and policy initiatives. \newline
It’s no secret that the world of healthcare has changed over the last hundred years. Things have improved; new discoveries have been made. What was once thought of as impossible and unthinkable may have now even become the new norm. As technology constantly improves, it has affected all types of industries, including healthcare.

\section{Evolution Of Modern Healthcare System}
\textbf{\emph{Technology}} plays a big part in the evolution of the human healthcare system. Wherever you look in hospitals, there are various tech gadgets and contrivances that does work in as simple as data encoding to actually supporting human life and helping those in near-death circumstances bounce back again. Computers are everywhere and healthcare professionals use it non-stop in the delivery of vital health care and life-saving procedures. It is even possible that in the near future, artificial intelligence will also be used and definitely transform the entire concept of patient care. We can only imagine what happens until that time comes and continue to marvel at all the big and little things technology accomplishes in the healthcare setting. \newline \\
Compare medical care right now, to one hundred years ago, and you’ll see so much technology has changed. In fact, compare it to the year 2000 and things are still dramatically different. Many of the medical gadgets that may have been in use before are already considered obsolete now.\newline
There is technology in hospitals that were unheard of many moons ago. All you have to do is simply browse through photos and videos of hospitals in the past, and you’ll be left in awe wondering how they were even able to survive without the medical equipment that is present today. \newline 
\textbf{\emph{From counting beds to calculating bytes.}}
While fee-for-service revenue and value-based reimbursement, health systems are building technology around populations, rather than trying to bring patients into the hospital.\newline
The simple fact is, technology advances so quickly, and there are so many new things being created and discovered. As a result, industries will benefit, and things are going to change and improve. Healthcare is one of the industries that has benefited and changed the most.\newline \\

\textbf{\emph{Communication}} has also been improved because of technology.\newline
Hospitals can communicate with one another a lot better, and share medical records. Rather than keeping and transporting physical documents, which could take quite some time, files and results can now be sent through the e-mail and other digital means.Records are now stored online, so they’re safer and more secure. You don’t have to worry about files getting permanently lost due to fire and other hazards.\newline

Overall ,Technology has had a huge impact on how healthcare has changed over the last hundred years. \newline \\
The \textbf{\emph{Healthcare delivery system}} has massively and rapidly evolved over the past ten years. Medical delivery sites have evolved from acute \textbf{\emph{inpatient}} care to \textbf{\emph{outpatient}} health services such as clinics, ambulatory services, surgical care centers, and rehabilitation centers. Midlevel healthcare practitioners' emergence in the delivery system such as certified midwives, medical assistants, physician assistants, and nurses. Medical diagnosis has advanced with new modes such as genetic testing. The delivery system has changed with the adoption of managed care to reduce medical services costs, with maximum utilization and quality. The massive explosion of biotechnology in the past years has expanded the products used in healthcare delivery. The delivery system shifted to value-based health care.\newline \\
The field of \textbf{\emph{Medical research}} has changed health care in so many ways. A century ago, we had very little research compared to what we have now . Think about how many cures for diseases have been found in the last century? Or, think about how many new drugs and treatments have been developed thanks to clinical pharmacology.  \newline \\
What used to take days, in terms of results and diagnostic tests, can now be completed within hours. Surgeries use more state-of-the-art machines and equipment. Diseases that were once untreatable are now considered common. These are only a few of the many changes that the healthcare industry has experienced.

\section{Conclusion}
Societal changes and scientific advances throughout history have brought about enormous improvements in the achievement of health. Today, an optimized level of “health,” whatever the definition might be, is fathomable and achievable if given unlimited resources. The problem lies in that resources are not unlimited, and are in fact disproportionately allocated between demographic groups. A value-based system, designed to provide a high quality of healthcare for the lowest cost, is a solution to the growing crisis of healthcare systems. A major problem with value-based care, however, is that these health outcomes are subjective and determined by individual patient needs and values. A new definition of “health,” which incorporates a description of well-being, specific patient needs, and the organizational, value-based system required to satisfy those needs, is now necessary.




\end{document}